\documentclass{article}

\usepackage[T1]{fontenc}      % un second package
\usepackage[francais]{babel}  % un troisième package
\usepackage[utf8]{inputenc}
\usepackage{amsmath,amsfonts,amssymb}
\usepackage{amsthm}
\usepackage{color}

\begin{document}


\section{Topologie }

\subsection{Rappels }



\theoremstyle{plain}
\newtheorem{theo} {\textcolor{red}{ Propriété }}[subsection]
\begin{theo}
$\mathbb{R}$ est archimédien, c'est-à-dire :\textcolor{blue}{ \[\forall (x,y) \in \mathbb{R}^{2} \ x> 0 \Rightarrow \exists n \in \mathbb{N} \   nx > y \]}

\end{theo}

\begin{theo}
$\mathbb{Q}$ est dense dans $\mathbb{R}$ ie :\textcolor{blue}{ \[ \forall (x,y) \in \mathbb{R}^{2} \ \  x<y \  \Rightarrow \    \exists q \in \mathbb{Q} \ x<q<y \]}

\end{theo}

\begin{theo}
Soit H un sous groupe de ($\mathbb{R}$,+), alors :\textcolor{blue}{
\[ \exists a \in \mathbb{R^{+}} \ \ H= a\mathbb{Z} \   ou \ H \ est \ dense \ dans \ \mathbb{R}\]}

\end{theo}
\subsection{Convexité }

\begin{theo}
\[ Soit \ E \ un\  \mathbb{K}-espace \  vectoriel.\  Soit\  A \subset E\]
 
On dit que A est une partie convexe de E lorsque :\textcolor{blue}{
\[ \forall (x,y) \in A^{2} \ \exists \lambda \in [0,1] \ \ \lambda x + (1- \lambda) y \in A \]}
\end{theo}


\begin{theo}
Il y a équivalence entre : \textcolor{blue}{
\[(i) \   A \ est \ convexe \]
\[(ii) \ \ \forall (x_{i})_{i \in \mathbb{N}} \in A^{n} \ \ 
\forall (\lambda_{i})_{i \in \mathbb{N}} \in [0,1]^{n}
\ \ \sum_{i=0}^{n}\lambda_{i} = 0 \ \Rightarrow \ \sum_{i=0}^{n} \lambda_{i}x_{i} \in A
    \]}
\end{theo}

\begin{theo}
\[Une\ intersection \ quelconque \ de\ parties\ convexes\ est\ convexe. \]
\end{theo}

\begin{theo}
\[Les\ boules\ sont\ des\ parties\ convexes. \]
\end{theo}

\begin{theo}
\textcolor{blue}{
\[A\ est\ bornee \ \Leftrightarrow \ \exists r \in \mathbb{R}^{+*} \ \ A \subset B(0,r) \]}

\end{theo}
\subsection{Valeurs d'adhérence }
\begin{theo}
Une suite a une seule valeur d'adhérence : sa limite. \\
Par contraposée, une suite qui possède deux valeurs d'adhérence ou plus diverge.

\end{theo}

\newtheorem{exo} {\textcolor{black}{ Exercice }}[subsection]
\newtheorem{defi} {\textcolor{green}{ Définition }}[subsection]

\begin{exo}
Si $(u_{n})_{n \in \mathbb{N}} \in\mathbb{K}^{\mathbb{N}}$  est bornée et possède une unique valeur d'adhérence, alors elle converge.
Démonstration par construction explicite de $\phi$.

\end{exo}


\begin{theo}
Soit $(u_{n})_{n \in \mathbb{N}} \in{E}^{\mathbb{N}}$.
Soit a\ $\in{E}$.\\
Il y a équivalence entre :
\textcolor{blue}{
\[ (i)\ a\ est\ valeur\ d'aderence\ de\ (u_{n}). \]
\[(ii)\ \forall \epsilon>0 \ \forall N \in \mathbb{N} \ \exists n \in \mathbb{N}\ \ (n\geq N\ \ et\ u_{n}\in\ \ B(a,\epsilon)\]}

\end{theo}
\subsection{Domination, équivalence }
\begin{theo}
Soit E un $\mathbb{K}$ espace vectoriel ,soit $ N_{1}$ et $N_{2}$ deux normes sur E.\\
Il y a équivalence entre :
\textcolor{blue}{
\[(i) \  N_{2} \  domine \  N_{1}\]
\[(ii) \ \exists r>0 \ B_{N_{2}}(0,1)\subset B_{N_{1}}(0,r)\]}

\end{theo}

\begin{theo}[Caractérisation séquentielle]
Soit E un $\mathbb{K}$ espace vectoriel ,soit $ N_{1}$ et $N_{2}$ deux normes sur E.\\
Il y a équivalence entre :
\textcolor{blue}{
\[(i) \  N_{2} \  domine \  N_{1}\]
\[(ii) \ Toute\ suite\ convergeant\ vers\ 0\ au\ sens\ de\ N_{2}\ converge\ au\ sens\ de\ N_{1}. \]}
\end{theo}

\subsection{Topologie }


\begin{defi}

Soit a $\in$E . Soit V $\subset$E.\\
On dit que V est un voisinage de a lorsqu'il existe r>0 tel que B(a,r) $\subset$ V.


\end{defi}


\subsubsection{Ouverts et Intérieur}
\begin{theo}
\textcolor{blue}{
\[(1) \  E\ et\ \emptyset \ sont\ des\ ouverts. \]
\[(2) \ Une\ union\ quelconque\ d'ouverts\ est\ un\ ouvert. \]
\[(3) \  Une\ intersection\ finie\ d'ouverts\ est\ un\ ouvert. \]
\[ (4) \ A=\overset{o}{A}\ \Leftrightarrow \ A\ est\ un\ ouvert.\]
\[(5) \ A\subset B \Rightarrow \ \overset{o}{A}\subset\overset{o}{B} \]
\[(6) \ \overset{o}{A\cap B} \ = \ \overset{o}{A}\cap \overset{o}{B} \]
\[(7) \ \overset{o}{A}\cup \overset{o}{B} \ \subset \ \overset{o}{A\cup B} \]}

\end{theo}




\begin{theo}
 Soit A $\subset$E. Soit a $\in$A .\\
 Il y a équivalence entre :
 \textcolor{blue}{
\[ (i) \ a\in \overset{o}{A} \]
\[(ii) \ il\ existe\ U\ ouvert\ de\ E\ tel\ que\ a\in U\ et\ U\subset A\]
\[(iii) \ il\ existe\ r>0\ tel\ que\ B(a,r)\subset A\]}


\end{theo}

\subsubsection{Fermés et Adhérence}

\begin{theo}
\textcolor{blue}{
\[(1) \  E\ et\ \emptyset \ sont\ des\ fermes. \]
\[(2) \ Une\ union\ finie\ de\ ferme\ est\ un\ ferme. \]
\[(3) \  Une\ intersection\ quelconque\ de\ fermes\ est\ un\ ferme. \]
\[ (4) \ A=\overset{-}{A}\ \Leftrightarrow \ A\ est\ un\ ferme.\]
\[(5) \ A\subset B \Rightarrow \ \overset{-}{A}\subset\overset{-}{B} \]
\[(6) \ \overset{-}{A\cap B} \ = \ \overset{-}{A}\cap \overset{-}{B} \]
\[(7) \ \overset{-}{A}\cup \overset{-}{B} \ = \ \overset{-}{A\cup B} \]}

\end{theo}

\begin{theo}
 Soit A $\subset$E. Soit a $\in$E .\\
 Il y a équivalence entre :
 \textcolor{blue}{
\[ (i) \ a\in \overset{-}{A} \]
\[(ii) \ \forall \ V\in V(a) \ \ V\cap A \ne \emptyset\]
\[(iii) \ \forall \ r>0 \ \ B_{f,o}(a,r)\cap A \ne \emptyset\]}


\end{theo}

\subsubsection{Densité etc}



\begin{theo}

 Soit A $\subset$E. 
 \textcolor{blue}{
 \[(i) \  \overset{-}{A}\ =\ \complement_{E}^{\overset{o}{\complement_{E}^{A}}} \]
  \[(ii) \  \complement_{E}^{\overset{-}{A}}\ =\ \overset{o}{\complement_{E}^{A}} \]
 }


\end{theo}

\begin{defi}

Soit A $\subset$E . \\
On dit que A est dense dans E lorsque $\overset{-}{A}$=E .

\end{defi}


\begin{theo}

 Soit A $\subset$E. \\
 \textcolor{blue}{
 A est dense dans E $\leftrightarrow$ $\forall \  r>0\ \forall a \in E\ B(a,r)\cap A \ne \emptyset$  
 }


\end{theo}

\begin{defi}

Soit A $\subset$E . \\
On appelle frontière de A la partie fr(A)=$\overset{-}{A} \setminus \overset{o}{A}$

\end{defi}

\begin{theo}

 fr(a) est un fermé.

\end{theo}







\end{document}



